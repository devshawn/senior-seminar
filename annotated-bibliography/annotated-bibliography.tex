\documentclass [11pt]{article}
\usepackage[margin=1in]{geometry}
\usepackage{hyperref}
\hypersetup{
    colorlinks=true,
    linkcolor=blue,
    filecolor=magenta,
    urlcolor=blue
}

\urlstyle{same}


\title{Vertex Coloring\\\medskip Annotated Bibliography}
\author{Shawn Seymour\\University of Minnesota Morris}
\date{Febuary 7, 2017}

\begin{document}
\maketitle
\section*{Summary}
The vertex coloring problem (VCP) is one of the most researched problems in graph theory. A proper vertex coloring is an assignment of colors, or labels, to each vertex in a graph \(G = (V, E)\) such that no edge connects to vertices of the same color. The VCP aims to find the chromatic number, denoted \(\chi(G)\), the minimum number of colors needed to properly color a graph as defined above. There are many heuristics to solve this NP-hard problem as well as ways to estimate and put bounds on the chromatic number, which I will explore in this paper.

\nocite{*}
\bibliographystyle{IEEEannot}
\bibliography{../references.bib}
\end{document}
