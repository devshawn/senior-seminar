\documentclass{sig-alternate}
\usepackage{color}
\usepackage[colorinlistoftodos]{todonotes}

%%%%% Uncomment the following line and comment out the previous one
%%%%% to remove all comments
%%%%% NOTE: comments still occupy a line even if invisible;
%%%%% Don't write them as a separate paragraph
%\newcommand{\mycomment}[1]{}

\begin{document}

\conferenceinfo{UMM CSci Senior Seminar Conference, May 2017}{Morris, MN}

\title{Conflict-Free Vertex Coloring Of Planar Graphs}

\numberofauthors{1}

\author{
\alignauthor
Shawn S. Seymour\\
	\affaddr{Division of Science and Mathematics}\\
	\affaddr{University of Minnesota, Morris}\\
	\affaddr{Morris, Minnesota, USA 56267}\\
	\email{seymo079@morris.umn.edu}
}

\maketitle
\begin{abstract}
Abstract goes here...
\end{abstract}

\section{Introduction}
\label{sec:introduction}

Consider the map of the 48 contiguous states in the United States of America. Suppose we would like to color each state such that no two states that share a boundary have the same color. This problem can be modeled with a mathematical structure called a \emph{graph}. We can represent each state with a \emph{vertex} and represent a boundary between two states with an \emph{edge}. This map is an example of a planar graph, i.e. none of the edges cross when drawn on a plane.

This is a famous example of the \emph{vertex coloring} problem and one of many graph coloring problems. The vertex coloring problem aims to find the minimum amount of colors needed to color a graph such that no two adjacent vertices are colored with the same color. While some problems are relatively easy to solve, the vertex coloring problem is one of the most computationally complex problems in computer science and mathematics. The vertex coloring problem has many real-world applications such as exam timetabling, register allocation, and the scheduling of taxis.

The \emph{conflict-free coloring} problem is a relaxed variation of the vertex coloring problem. The conflict-free coloring problem does not aim to color every vertex such that it is not connected to a vertex of the same color. Rather, it aims to color \emph{enough} vertices such that for every vertex, it is connected to at least one colored vertex.


\section{Background}
\label{sec:background}
To understand the problem, the algorithms to solve it, and its results, we must first understand some graph theory, some computational complexity theory, and the precise definitions of the vertex coloring problem and the conflict-free coloring problem.

\subsection{Graph Theory}
\label{sec:graphtheory}

A graph, denoted $G=(V,E)$, is an ordered pair of two sets: a set of vertices $V$ and a set of edges $E$. Each edge consists of a pair of vertices from $V$. For example, $(u,w) \in E$ is an edge connecting vertices $u$ and $w$ where $u,w \in V$. Vertices are adjacent if they are connected by an edge. A graph can be non-technically described as a set of points (vertices) with lines (edges) connecting them. A simple graph is an undirected graph where each edge connects two different vertices and no two edges connect the same pair of vertices. This means a vertex cannot have a loop, i.e. an edge from a vertex to itself. \cite{bondy1976graph}

The neighborhood of a vertex, denoted $N_G(v)$, is a set of all vertices adjacent to $v$. A \emph{closed} neighborhood consists of all vertices adjacent to $v$ and $v$ itself. Unless stated otherwise, we will assume when talking about neighborhoods that they are closed neighborhoods. The degree of a vertex, denoted $d_G(v)$, is the number of vertices adjacent to $v$. The maximum degree of a graph $G$, denoted $\Delta(G)$, is the maximum degree of its vertices.


\subsection{Graph Coloring}
\label{sec:coloring}
A vertex coloring is an assignment of colors to each vertex of a graph $G$ such that no adjacent vertices share the same color. Mathematically, it can be described as a function $f : V \rightarrow S = \{1, 2, \dots, k\}$ such that $\forall u,w \in V$, if $(u,w) \in E$, then $f(u) \neq f(w)$. The \emph{chromatic number} of $G$, denoted $\chi(G)$, is the minimum number of colors, i.e. $|S|$, needed to properly color $G$.


\subsection{Complexity Theory}
\label{sec:complexitytheory}


\section{C-F Coloring of General Graphs}

\section{C-F Coloring of Planar Graphs}


\section{Complexity of Coloring C-F Graphs}


\subsection{General Graphs}


\subsection{Planar Graphs}


\section{Conclusion}
\label{sec:conclusion}

\section{Acknowledgments}
\cite{abel2017three}


\bibliographystyle{abbrv}
\bibliography{../references}


\end{document}
