% This is a sample document using the University of Minnesota, Morris, Computer Science
% Senior Seminar modification of the ACM sig-alternate style. Much of this content is taken
% directly from the ACM sample document illustrating the use of the sig-alternate class. Certain
% parts that we never use have been removed to simplify the example, and a few additional
% components have been added.

% See https://github.com/UMM-CSci/Senior_seminar_templates for more info and to make
% suggestions and corrections.

\documentclass{sig-alternate}
\usepackage{color}
\usepackage[colorinlistoftodos]{todonotes}

%%%%% Uncomment the following line and comment out the previous one
%%%%% to remove all comments
%%%%% NOTE: comments still occupy a line even if invisible;
%%%%% Don't write them as a separate paragraph
%\newcommand{\mycomment}[1]{}

\begin{document}

% --- Author Metadata here ---
%%% REMEMBER TO CHANGE THE SEMESTER AND YEAR AS NEEDED
\conferenceinfo{UMM CSci Senior Seminar Conference, May 2017}{Morris, MN}

\title{Conflict-Free Vertex Coloring Of Planar Graphs}

\numberofauthors{1}

\author{
\alignauthor
Shawn S. Seymour\\
	\affaddr{Division of Science and Mathematics}\\
	\affaddr{University of Minnesota, Morris}\\
	\affaddr{Morris, Minnesota, USA 56267}\\
	\email{seymo079@morris.umn.edu}
}

\maketitle
\begin{abstract}
Abstract goes here...
\end{abstract}

\section*{KEY POINTS}
The goal of this research is to discover and explain properties and results of conflict-free coloring of planar graphs. The conflict-free coloring problem is a variation of the classical vertex coloring problem (VCP). The VCP aims to color a graph \(G = (V, E)\) such that no two adjacent vertices receive the same color. This is typically shown in the form of the \emph{k-colorability} problem: a graph is \emph{k-colorable} if it can be colored using $k$ or less colors. For example, the map of the 48 contiguous states can be colored with 4 colors and is thus \emph{4-colorable}. This is shown in the famous four-color theorem. It cannot be colored with 3 colors, and is thus \emph{not 3-colorable}.

The minimum amount of colors needed to color a graph is called the \emph{chromatic number}. It is heavily studied in graph theory and it is NP-hard to determine the chromatic number of a graph. This property makes it an interesting problem to study as there are currently no known polynomial time algorithms to solve the VCP.

A conflict-free coloring is a variation of the VCP. Each vertex must be assigned a color such that for every vertex $v$, there is at least one color appearing exactly once in the neighborhood of $v$, i.e. vertices adjacenct to $v$.

\section{Introduction}
\label{sec:introduction}
Explain why graph coloring, specifically vertex coloring, is interesting and how it can be used to solve many real-world applications. Describe how it can be one of the most difficult problems to solve efficiently. Bring up the example of coloring the 48 contiguous states, an example of a planar graph. Introduce conflict-free coloring as a variant of the classical vertex coloring problem (VCP). 

\section{Background}
\label{sec:background}
Give some of the needed background regarding graph theory, complexity theory (i.e. NP-complete, NP-hard, etc), and the classical vertex coloring problem.

\subsection{Graphs}
\label{sec:graphs}
Define what a graph, \(G = (V, E)\), is. Define terms that will be used throughout the paper dealing with graphs such as nodes, edges, neighborhoods, etc.

\subsection{Vertex Coloring Problem (VCP)}
\label{sec:vcp}
Explain the classical vertex coloring problem and why it is interesting. Give required background so conflict-free colorings will make sense later.

\subsection{Complexity of the VCP}
\label{sec:complexitytheory}

\section{Conclusions}

\section{Acknowledgments}
Give thanks to my mom, etc.


\cite{bondy1976graph}


\bibliographystyle{abbrv}
\bibliography{../references}


\end{document}
