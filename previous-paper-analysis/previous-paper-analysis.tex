\documentclass[12pt,letterpaper]{article}
\usepackage[utf8]{inputenc}
\usepackage[margin=1in]{geometry}
\usepackage[doublespacing]{setspace}

\title{Previous Paper Analysis}
\date{January 24, 2017}
\author{Shawn Seymour}

\begin{document}

\maketitle

I read Molly Grove's paper, \emph{Heuristics for the Generalized Traveling Salesman Problem}, from the Fall 2015 senior seminar conference \cite{grove}. Her paper details three heuristics, also known as approximation algorithms, for the generalized traveling salesman problem (GTSP). The paper introduces the topic and gives background information on graph theory, computational complexity, and the mathematical definition of the original traveling salesman problem (TSP) and the GTSP.

The structure of the paper is very clear. Each section has a clear point; The paper starts with an introduction, gives background information we'll need to know to analyze the heuristics, and then details 3 different heuristics for the GTSP. Each heuristic is broken into clear parts (as the given heuristics are combinations of different algorithms) and each part of the heuristic is described as a subsection.

For the most part, one paper per heuristic is used to analyze the heuristic and its algorithms. A few sources are used to provide background information on the problem, on computational complexity, and on certain algorithms relied upon in the presented heuristics.

The sources used by Grove are almost all recent papers at the time the paper was written. The key sources for the heuristics were all within 5 years and most of her supporting sources were within 7 years. The only source that stood out as being old was her reference [4] written by Papadimitriou and Steiglitz, which was published in 1982.

Grove presents the material with diagrams and explanations of the algorithms that each heuristic uses. As the TSP and GTSP are graph problems, she uses helpful diagrams of nodes and edges and shows how the heuristics work on the given graph. These are very helpful to understand the different clustering techniques that are used in a couple of the given heuristics.

The paper was fairly easy to understand although some of the specifics got complicated. There were some subsections that were quite mathematically heavy and a few of these examples or formulas could have been explained a bit simpler for a data-structures level student. I believe the structure for presenting the material was quite good but a few of the specifics could have been explained in simpler terms.

I really liked the paper's structure and presentation of the different heuristics. Each heuristic is broken up based on the mathematics behind it and the algorithms and techniques used which is how I would do it. For some of the explanations, I would have simplified some things or provided clearer examples of certain techniques presented in the paper.
\newline

\begin{thebibliography}{9}

\bibitem{grove}
	Molly Grove,
	\emph{Heuristics for the Generalized Traveling Salesman Problem},
	UMM CSci Senior Seminar Conference, Fall 2015, Morris, MN.

\end{thebibliography}

\end{document}
